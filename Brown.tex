\documentclass[11pt]{beamer}
\usepackage{listings} % Include the listings-package
\usepackage[T1]{fontenc}
\usepackage[utf8]{inputenc}
\usepackage[english]{babel}
\usepackage{amsmath}
\usepackage{amssymb, amsfonts, latexsym, cancel}
\usepackage{float}
\usepackage{graphicx}
\usepackage{epstopdf}
\usepackage{subfigure}
\usepackage{hyperref}
%\usepackage{authblk}
\usepackage{blindtext}
\usepackage{booktabs} % Allows the use of \toprule, 
\usepackage{filecontents}
\usepackage{courier} %% Sets font for listing as Courier.
\usepackage{listings}
\usepackage{ragged2e}
%\usepackage{listings, xcolor}
\lstset{
tabsize = 2, %% set tab space width
showstringspaces = false, %% prevent space marking in strings, string is defined as the text that is generally printed directly to the console
numbers = left, %% display line numbers on the left
commentstyle = \color{green}, %% set comment color
keywordstyle = \color{blue}, %% set keyword color
stringstyle = \color{red}, %% set string color
rulecolor = \color{black}, %% set frame color to avoid being affected by text color
basicstyle = \small \ttfamily , %% set listing font and size
breaklines = true, %% enable line breaking
numberstyle = \tiny,
}
\usepackage{caption}
\DeclareCaptionFont{white}{\color{white}}
\DeclareCaptionFormat{listing}{\colorbox{gray}{\parbox{\textwidth}{#1#2#3}}}
\captionsetup[lstlisting]{format=listing,labelfont=white,textfont=white}
\definecolor{urlColor}{rgb}{0.06, 0.3, 0.57}
\definecolor{linkColor}{rgb}{0.57, 0.0, 0.04}
\definecolor{fileColor}{rgb}{0.0, 0.26, 0.26}
\hypersetup{
    colorlinks=true,
    linkcolor=linkColor,
    filecolor=fileColor,      
    urlcolor=urlColor,
}
\urlstyle{same}
\setbeamercovered{transparent}
%\usetheme{Boadilla}
\usetheme{CambridgeUS}
%\usetheme{Berkeley}
%\usetheme{Warsaw}
%\usetheme{Madrid}

\title[Introducción]{\bf\Huge C. Marlin "Lin" Brown}
\subtitle{Interacción Humano Computador}

\author[rescobedoq]
{
    INTEGRANTES:
	\newline Gabriel Callo Condori 
	\newline José Rodríguez Mercado
	\newline José Miguel Vera Mamani
	\newline Rosario Lohana Alegre Linares
}
\institute[UNSA]
{
\inst{1}% 
System Engineering School\\
System Engineering and Informatic Department\\
Production and Services Faculty\\
San Agustin National University of Arequipa
}

\date[2020-09-09]{\scriptsize{2020-09-15}}
\logo{\includegraphics[width=3.0cm]{logo_unsa.jpg}}
%\titlegraphic{\includegraphics[width=4.5cm]{logo_unsa.jpg}}

\begin{document}

\begin{frame}
\titlepage
\end{frame}

\begin{frame}
\frametitle{Content}
\tableofcontents
\end{frame}

\section{Conociendo a Brown}
\begin{frame}
\frametitle{Conociendo a Brown}
\begin{itemize}
\item Trabajó con tecnologías emergentes en Xerox Palo Alto Research Center.
\begin{figure}
  \centering
  \includegraphics[width=0.3\textwidth]{palo.jpg} 
\end{figure}

\item Obtuvo un Doctorado del Instituto de Tecnología de Georgia en Psicología de la Ingeniería de Factores Humanos.
\item Enseñó en Gestión de Sistemas y Factores Humanos.

\end{itemize}
\end{frame}

\section{Aportes}
\begin{frame}
\frametitle{Aportes}
\begin{minipage}[c]{0.4\textwidth} 
\includegraphics[width=4.5cm, height=6cm]{libro de marlin.jpg} 
\end{minipage}
\begin{minipage}[c]{0.55\textwidth} 
{\bf Pautas de diseño de la Interfaz Humano-Computador:}
\newline Es un libro clásico que contiene cientos de pautas prácticas y reglas generales para ayudar a los diseñadores de software a desarrollar interfaces Humano-Computadora orientadas al usuario.
Se puede encontrar un conjunto de sugerencias para ayudar a los diseñadores de la interfaz a comprender la relación entre los sistemas informáticos y sus usuarios.
\end{minipage} 
\end{frame}

\section{}
\begin{frame}
Estas sugerencias y pautas se extraen de: 
\begin{itemize}
\item Evidencia de experimentos
\item Predicciones de las teorías del desempeño humano
\item Principios de psicología cognitiva
\item Principios de diseño ergonómico
\item Evidencia recopilada a través de la experiencia en ingeniería
\end{itemize}

\justifying \vspace{5mm} Muchas pautas van acompañadas de ejemplos. {\bf Un capítulo final presenta una estrategia para implementar los principios de diseño de la interfaz Humano-Computadora.}

\end{frame}


\section{References}
%References frame
\begin{frame}
\frametitle{References}
\begin{itemize}
\item Introduction - xiii, Jhonson J. (2014). Designing with the Mind in mind. 2nd. edition.
\item Introduction, C. Marlin Brown (1999). Human–computer interface design guidelines.
\end{itemize}
\end{frame}

\end{document}